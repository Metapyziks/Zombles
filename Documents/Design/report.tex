\documentclass[12pt,a4paper]{article}
\usepackage{times}
\usepackage[english]{babel}
\usepackage{durhampaper}
\usepackage{harvard}

\citationmode{abbr}
\bibliographystyle{agsm}

\title{Real-Time Human Directed Multi-Agent Task Planning in a Simulated Hazardous Environment}
\student{James King}
\supervisor{Magnus Bordewich}
\degree{BSc Computer Science}

\date{\today}

\begin{document}

\maketitle

\begin{abstract}

{\bf Context:} Designing the core autonomous behaviour for characters in a video game where those characters are expected by the player to perform tasks for themselves poses many interesting challenges. This paper explores the tasks faced when developing a capable and somewhat convincing set of Artificial Intelligence routines within a simulated zombie epidemic real-time strategy game with player allocated goals.

{\bf Aims:} As the game player doesn't have direct control over the characters, but instead can assign tasks, algorithms to allow the characters to plan and implement those tasks in a cooperative and efficient manor will need to be designed. These algorithms must not be computationally expensive to allow for many agents to act in real-time, and also express superficially convincing human-like behaviours.

{\bf Method:} At least two conceptually distinct designs for character behaviour will be designed, and additionally several slight variations of each. These will all be compared in terms of the ability for characters to achieve assigned tasks, avoid threats, and system resource usage. A hybrid between the main designs may be explored if one is not universally better than the other.

{\bf Proposed Solution:} A Subsumptive architecture will initially be explored, followed by a Belief-Desires-Intentions model. Following this, each approach will be adapted to experiment with different character behaviours and strategies. The nature of the game environment may be altered to accommodate improvements necessitated by the implementations of each model. 
\end{abstract}

\begin{keywords}
Put a few keywords here.
\end{keywords}

\section{Introduction}
This section briefly introduces the project, the research question you are addressing.  Do not change the font sizes or line spacing in order to put in more text.

Note that the whole report, including the references, should not be longer than 12 pages in length (there is no penalty for short papers if the required content is included). There should be at least 5 referenced papers.

\section{Design}

This section presents the proposed solutions of the problems in detail. The design details should all be placed in this section. You may create a number of subsections, each focusing on one issue.

This section should be up to 8 pages in length.
The rest of this section shows the formats of subsections as well as some general formatting information.  You should also consult the Word template.

\subsection{Main Text}

The font used for the main text should be Times New Roman (Times) and the font size should be 12.  The first line of all paragraphs should be indented by 0.25in, except for the first paragraph of each section, subsection, subsubsection etc. (the paragraph immediately after the header) where no indentation is needed.

\subsection{Figures and Tables}
In general, figures and tables should not appear before they are cited.  Place figure captions below the figures; place table titles above the tables.  If your figure has two parts, for example, include the labels ``(a)'' and ``(b)'' as part of the artwork.  Please verify that figures and tables you mention in the text actually exist.  make sure that all tables and figures are numbered as shown in Table \ref{units} and Figure 1.
%sort out your own preferred means of inserting figures

\begin{table}[htb]
\centering
\caption{UNITS FOR MAGNETIC PROPERTIES}
\vspace*{6pt}
\label{units}
\begin{tabular}{ccc}\hline\hline
Symbol & Quantity & Conversion from Gaussian \\ \hline
\end{tabular}
\end{table}

\subsection{References}

The list of cited references should appear at the end of the report, ordered alphabetically by the surnames of the first authors.  The default style for references cited in the main text is the  Harvard (author, date) format.  When citing a section in a book, please give the relevant page numbers, as in \cite[p293]{budgen}.  When citing, where there are either one or two authors, use the names, but if there are more than two, give the first one and use ``et al.'' as in  , except where this would be ambiguous, in which case use all author names.

You need to give all authors' names in each reference.  Do not use ``et al.'' unless there are more than five authors.  Papers that have not been published should be cited as ``unpublished'' \cite{euther}.  Papers that have been submitted or accepted for publication should be cited as ``submitted for publication'' as in \cite{futher} .  You can also cite using just the year when the author's name appears in the text, as in ``but according to Futher \citeyear{futher}, we \dots''.  Where an authors has more than one publication in a year, add `a', `b' etc. after the year.




\bibliography{projectpaper}


\end{document}