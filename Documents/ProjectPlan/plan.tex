\documentclass[a4paper,11pt]{article}

\usepackage{lipsum}
\usepackage{listings}
\usepackage{graphicx}
\usepackage[margin=20mm]{geometry}

\title{\centering Project Plan}
\date{\today}
\author{James King}

\begin{document}
\section{Minimum Deliverables}
\subsection{Prepare existing codebase to be expanded upon and fix existing issues}
    \begin{itemize}
    \item Convert source files to adhere to current coding conventions
    \item Make use of .NET 4.5 features where applicable
    \item Overhaul world generation system to require less repetition of code
    \item Test and fix ray tracing of a moving collision hull
    \end{itemize}
\subsection{Implement new pathfinding system and threat avoidance}
    \begin{itemize}
    \item Place path vertices at significant positions within each block of
        the city for use when pathing between two nearby positions
    \item Implement an abstracted higher level path finding system for finding
        general paths between two remote locations, using the intersections of
        roads in the city as vertices
    \item Allocate a limited amount of time for each agent to calculate paths
        per frame, and store the partial result to be continued next frame if
        required
    \item Add geometry \emph{hint} nodes to mark features of the world that can
        be exploited when agents are avoiding threats, for example hints for
        doorways and room corners
    \item Rewrite agent threat avoidance to take into account geometry hints
    \item Test the new path finding system by instructing agents to path to
        randomly selected locations in the world
    \end{itemize}
\subsection{Analyse system scalability}
    \begin{itemize}
    \item Evaluate frame time and memory allocated for a range of world
        sizes, from trivial to the largest possible
    \item For each world size, record the above metrics for varying numbers
        of agents, from none to the limit that increases frame time to be
        larger than some target, with path finding disabled
    \item With some configuration of the above two parameters, experiment
        with the amount of time allocated for agents to find paths to
        analyse frame times
    \item Experiment with limiting the range of agents path finding node
        exploration to currently or previously visible tiles, and then
        for currently occupied or previously occupied blocks, and compare
        performance and memory used
    \end{itemize}

\section{Intermediate Objectives}
\subsection{Improve and extend simulation environment}
    \begin{itemize}
    \item Add mesh entity rendering component for entities with 3D models
    \item Implement barricade entities that block off a tile that may be
        destroyed if attacked enough
    \item Add at least one decorative entity that will produce a resource
        (barricade material) when destroyed
    \item Improve world generation to produce several building types with
        different layouts and that contain varying amounts of resource
        dropping decorative entities
    \end{itemize}
\subsection{Implement subsumptive agent AI architecture}
    \begin{itemize}
    \item Build a wide variety of AI components to achieve different tasks
        or promote certain behaviours such as threat avoidance or
        exploration
    \item Organise the AI components into layers of abstraction, where
        more abstracted layers may choose between the decisions made by
        the preceding less abstracted one, or override them
    \item Compare survival rates and realism of different weightings
        for the subsumptive behaviour system, for example a configuration
        designed for flocking, and another where agents are solitary
    \end{itemize}
\subsection{Improve performance of path finding}
    \begin{itemize}
    \item Implement path caching to improve path finding performance
    \item Limit the capacity of the path cache, and remove infrequently
        used paths if this is exceeded
    \item Investigate the resulting cache hit and fault rates for cached
        path discarding methods
    \end{itemize}

\section{Advanced Objectives}
\subsection{Expand user interaction with the simulation}
    \begin{itemize}
    \item Add barricade material storage designations, and the ability
        for a player to place them
    \item Include method for a player to order a specified group of agents
        to travel to a given location
    \item Allow players to mark decorative entities that produce barricade
        material to be dismantled and scavenged from
    \item Implement barricade building designations that may be placed by
        a player
    \item Produce a method for players to toggle individual or groups of
        agents from being offensive or defensive
    \item Create new AI components to complete the new user designated tasks
    \end{itemize}
\subsection{Implement a BDI architecture for comparison}
    \begin{itemize}
    \item Construct an alternative agent AI architecture using a beliefs-
        desires-intentions model to perform the same tasks as the existing
        subsumptive one
    \item Attempt to build a rudimentary planning system on top of the new
        BDI oriented architecture to improve agent performance
    \item Compare the two architectures in terms of memory usage, frame
        times, behaviour and survival rates for a variety of different
        initial simulation states
    \item Assess the efficiency at which each architecture completes user
        assigned tasks in similar circumstances
    \end{itemize}
\end{document}
